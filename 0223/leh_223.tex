\documentclass[12pt]{article}
\usepackage[paper=letterpaper, left=1in, right=1in, top=1in, bottom=1in]{geometry}

\usepackage[parfill]{parskip}
\usepackage{amsmath}
\usepackage{graphicx}

\begin{document}
\thispagestyle{empty}

\begin{center}
{\large CS 310}\\
Assignment 223
\end{center}

\begin{flushright}
Hieu Le
\end{flushright}

1. Explain the changes you made to the program and
the algorithm you used to implement the traversal.

\textit{Answer:}

The function \texttt{levelorder()} implements a level order traversal of a tree 
with unrestricted number of children per node. The function takes in a pointer 
to the root of the tree as the sole argument and prints out the content at each 
of its node, level by level. It does so by iteratively visiting all nodes at 
every depth of the tree. This approach has the flavor of a bread-first search 
algorithm, and certainly so with the exception being that the search does not 
terminate until all the nodes have been visited.

The test \texttt{root == nullptr} checks whether the tree is empty. If the test
is true, \texttt{levelorder()} returns since there is nothing to print from the 
tree. The test is important since it allows the function to execute without any
error when an empty tree is input. Since the function is iterative in nature,
this test differs from the base case test for \texttt{nullptr} leaf nodes found 
in many recursive tree procedures.

The tree nodes yet to visit is stored in a queue object called \texttt{frontier}. The
use of a sequential data structure that supports first-in first-out element access
is crucial since we want to ensure the 'lefter' node at a smaller depth is visited
first. The use of a single queue throughout the function means that there is no 
distinguishing between nodes located at different depths. This is acceptable since
the content of the nodes is printed on the same line. 

\begin{verbatim}
tree_node * child = current->get_first_child();
while( child != nullptr )
{
    frontier.push( child );
    child = child->get_right_sibling();
}
\end{verbatim}

The nested while loop shown above iterates through all the children of 
\texttt{current} which is the node being visited and push them onto \texttt{frontier}. 
The outer while loop terminates when \texttt{frontier.empty() == true}, i.e all the 
nodes in the tree have been visited.

As is the case with most tree traversal algorithms, the running time complexity of 
this algorithm is $\Theta(n)$ where $n$ represents the number of nodes in the tree. 
The algorithm is linear since each node is pushed onto the queue exactly once.

2. For an arbitrary non-empty binary search tree, find a precise expression for the
number of \texttt{nullptr}. State and explain your answer.

\textit{Answer:}

\textit{Method 1. By observation.}

Let $T$ be a non-empty binary tree with $n$ nodes. Since each node contains $2$ pointers
to its left and right children, there are a total of $2n$ pointers in $T$. Each pointer 
either points to an interior node excluding the root of the tree or a \texttt{nullptr}.
As there are $n - 1$ such interior nodes, there must be $2n - (n - 1) = n + 1$ 
\texttt{nullptr} in $T$.

\textit{Method 2. By induction.}

Proposition: a non-empty binary tree with $n$ nodes has $n + 1$ \texttt{nullptr}. 

When $n = 1$, the proposition holds since the tree consists of a single node and
$2$ \texttt{nullptr}.

Assume that for an arbitrary positive integer $k$, the number of \texttt{nullptr} in
every non-empty binary tree of size $n \leq k$ is $n + 1$. Let $T$ be an arbitrary 
binary tree of size $n = k + 1$. Let $T_L$ and $T_R$ be the left and right subtrees 
of $T$ respectively. Subsequently, $T_L$ and $T_R$ cannot both be empty since there are at
least 2 nodes in $T$ and therefore at least 1 node in one of these subtrees. 

Case 1: Both subtrees are non-empty. Therefore, the number of \texttt{nullptr} of $T$
is the sum of the numbers of \texttt{nullptr} from each of these subtrees. Let $size(X)$ 
denotes the size of a tree $X$. By the inductive hypothesis, the number of \texttt{nullptr}
of $T$ is equal to:
\[
(size(T_L) + 1) + (size(T_R) + 1)
= (size(T_L) + size(T_R) + 1) + 1
= size(T) + 1
\]

Case 2: One of the subtrees is empty. Without loss of generality, assume that $T_L$ is empty.
Subsequently, the number of \texttt{nullptr} of $T$ is equal to:
\[
1 + (size(T_R) + 1)
= size(T) + 1
\]

By mathematical induction, it is therefore proven that the number of \texttt{nullptr} in a 
non-empty binary tree is 1 more than the number of nodes in the tree.

Since a binary search tree is itself a binary tree, the proven result can be applied to a
binary search tree.
\end{document}
